\chapter{Описание предметной области}
Диалоговая система – это компьютерная система, предназначенная для общения с человеком, с когерентной (последовательной) структурой \cite{jurafsky2017dialog}.

Первой системой подобного рода считается ELIZA разработанная в MIT в 1966 году. Она реализована при помощи набора правил ответа и поиска во входящей реплике(pattern matching). Такая архитектура называется  <<основанной на правилах>>. ELIZA имитировала диалог с психологом на основе правил активного слушания. В теории это должно было помочь пользователю снять психологическую напряженность или решить несложные психологические неурядицы.  В результате чего ELIZA является представителем «задаче ориентированных диалоговых систем» (Goal based). 

Значительный прогресс в диалоговых системах приходится на 1980-х годах, с ростом популярности и исследованности области машинного обучения и адаптивных алгоритмов. А так же применение новых подходов в компьютерной лингвистке (корпусная семантика, вероятностные грамматики)  позволило начать эксперименты с  алгоритмами машинного обучения в диалоговых системах. Открытие и имплементация символьных грамматик и представлений (Cache Language Models) \cite{li2018recurrent} позволило использовать слова не встречавшиеся раннее в корпусе, облегчив обработку и составление корпусов для обучения. 

Далее исследователи столкнулись с проблемами производительности вычислительных машин того времени и многие подходы до настоящего времени не могли быть использованы. 

В 1990-х в связи с ростом мощности вычислительных устройств стало возможным использование нейросетевого подхода и больших корпусов для обучения. 

До 2006 года об успехах компьютерной лингвистики и диалоговых систем знали только люди интересующиеся, пока компания IBM не устроила показательный матч между системой Watson и человеком в игре-викторине «Jeopardy!» (русская адаптация «Своя игра») , где человек не смог одержать верх над машиной. Тогда общественный интерес к диалоговым системам возрос. В дальнейшем это повлекло масштабные инвестиции в эту область и послужило причиной появления многих  популярных диалоговых агентов. 

В настоящее время большинство диалоговых систем нацелены на мобильные платформы и представляют собой неспециализированных диалоговых ассистентов. Такие ассистенты могут вести беседы на общие темы, использовать возможности устройства по запросу пользователя, а также производить поисковый запрос. 

Примерами таких неспециализированных систем могут служить: 
\begin{itemize}
\item Яндекс.Алиса – персональный помощник для мобильных платформ. Может вести диалог на отвлеченные темы на русском языке, осуществлять поисковые запросы, а так же запускать и взаимодействовать с приложениями установленными на устройстве. Имеет удобный режим диалога для взаимодействия. Сохраняет контекст предыдущих сообщений для формирования ответа. Может отвечать на вопросы вида «одну сущность назад», когда пользователь спрашивать о чем-то о чем говорил ранее \cite{fonarev2017riemannian}. 
\item Siry – персональный помощник от компании Apple. Может вести диалог на нескольких языках, осуществлять поисковые запросы. В целом направлен на взаимодействие с экосистемой устройств Apple. Не сохраняет контекст реплик. 
\end{itemize}
С ростом популярности облачных сервисов многие компании запустили так же сервисы для создания специализированных диалоговых систем на базе своих технологий. 
Примерами таких систем могут служить:
\begin{itemize}
    \item Dialogflow – платформа для создания диалоговых систем от Google. Разворачивается на базе облачного сервиса, поддерживает интеграцию в приложения. Не требует специализированных знаний для работы.
    \item Azure Bot Service – платформа для создания диалоговых систем от Microsoft. Разворачивается на базе облачного сервиса. Требует специальных знаний для оператора.
    \item Amazon Lex – платформа для создания диалоговых систем от Amazon. Разворачивается на базе облачного сервис. Требует специальных знаниий, поддерживает интеграцию с сервисами Amazon. Не доступен на русском языке. 
\end{itemize}
 
 
