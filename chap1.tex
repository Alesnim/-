\chapter{Описание предметной области}
Диалоговая система – это компьютерная система, предназначенная для общения с человеком, с когерентной (последовательной) структурой \cite{dialog_jur}.

Первой системой подобного рода считается ELIZA разработанная в MIT в 1966 году. Она реализована при помощи набора правил ответа и поиска во входящей реплике(pattern matching). Такая архитектура называется  "основанной на правилах". ELIZA имитировала диалог с психологом на основе правил активного слушания. В теории это должно было помочь пользователю снять психологическую напряженность или решить несложные психологические неурядицы.  В результате чего ELIZA является представителем «задаче ориентированных диалоговых систем» (Goal based). 