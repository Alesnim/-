\chapter{Реализация серверной части}
\section{Проектирование нейросетевой части приложения}
В рамках разделенной структуры сервисов для реализации необходимого функционала системы было принято решение написать функциональные интерфейсы для обученной нейросетевой модели. 

Подход к имплементации архитектуры сети позволял выбирать из множества различных решений. Многие из типовых архитектур позволяют строить не только отображения множества реплик в множества ответов, но также строят языковую модель и семантическое ядро текста. Полнофункциональные модели различных архитектур нейронных сетей имеют множетсво параметров, что в данном случае является избыточным. 

Было принято решение имплементировать реккурентные типы нейронных сетей, используя типовые архитектуры для генерации по примеру зарубежных коллег \cite{li2016deep, sharma2016natural,толкачев2019нейронное}. Использованные ими различные походы к язковому моделирования диалогов помогли с выбором арзитектуры ячеек сети и подготовкой набора данных для обучения. 

Для реализации была выбрана двунаправленная сеть долгой-короткой памяти (bidirectional Long-Short Time Memory network). Языком для реализации послужил Python 3 и набор необходимых библиотек для обучения нейронных сетей.

Для успешного обучения сети необходим качественный и хорошо подготовленный, обширный набор данных. Для целевого домена сети (психология) на момент написания выпускной квалификационной работы открытых наборов данных не существовало. Более того, существующие наборы данных на русском языке для более широкого семантического домена (диалоги) также не позволяли провести успешное обучение диалогового агента. 

Обучение сети проводилось на базе сервиса Colaboratory от компании Google. Сервис предоставляет бесплатные вычислительные мощности для исследователей и студентов в рамках учебных проектов. Colaboratory имеет удобный интерфейс и позволяет запускать фоновые задачи для обучений нейросетевых моделей. Испанские исследователи более подробно описывают функционал сервиса и проводят анализ и сравнение с аналогами \cite{carneiro2018performance}.

Перед обучением сети необходимо подготовить набор данных. Поскольку набор данных необходимо еще было собрать, было принято решение использовать в качестве источника диалогов между двумя коммуникационными агентами смешанный набор из выделенных частей субтитров с сайта Open Subtitles, как это было сделано в работах \cite{creutz2018open, arcan2016asistent} и выделенных из текстов пьес диалогов из открытых источников. 

Перед тем как добавить диалоги из субтитров необходимо было выделить их из файлов субтитров. Для этого был написан скрипт автоматизации экстракции содержимого субтитров. Скрипт извлекает и сортирует содержимое на токены, в данном случае токенами служат приложения. После чего они очищаются от непечатных и иных неподдерживаемых символов, стоп слов. Пример листинга скрипта представлен на листинге \ref{listing:3} 

\begin{listing}
\inputminted[breaklines, breakanywhere, linenos, fontsize=\small]{python}{source/prepare_subtitles.py}
\caption{Пример листинга}
\label{listing:3}
\end{listing}

