% Задаем класс документа 
\documentclass[a4paper,14pt]{extreport}

% Импорт шрифта  Times New Roman 
\usepackage{xltxtra}
\usepackage{color}
\usepackage{url}
\usepackage{subfigure}
\defaultfontfeatures{Ligatures=TeX,Mapping=tex-text}
\newfontfamily\cyrillicfontsf[Script=Cyrillic]{Arial}
\newfontfamily\cyrillicfonttt[Script=Cyrillic]{Courier New}
\setmainfont{Times New Roman}


\usepackage{minted}



% Шрифт для листингов кода
\setmonofont[BoldFont=font/Hack-Bold.ttf,
ItalicFont=font/Hack-Italic.ttf,
BoldItalicFont=font/Hack-BoldItalic.ttf
]{font/Hack-Regular.ttf}


% Импорт пакета для работы с киррилицей
\usepackage{polyglossia}
\setmainlanguage{russian}
\setotherlanguage{english}


%Что-то для Бабеля 
\setkeys{russian}{babelshorthands=true}

\usepackage{amsmath,amssymb,epsfig}
\usepackage{paralist}
% Отступ в 1,5 перед каждым абзацем
\usepackage{indentfirst}
% Параметры страницы 
\usepackage {geometry}
\geometry{left=2.5cm}       % левое поле
\geometry{right=1.0cm}      % правое поле
\geometry{top=2.0cm}        % верхнее поле
\geometry{bottom=2.0cm}     % нижнее поле
\usepackage{csquotes}

\usepackage{listings}

\lstset{ %
  numbers=left,
  captionpos= b,
  numbersep= 4mm,
  numberstyle= \small,
  showtabs =\true,
  frame = single,
  breaklines=true,
  extendedchars=\true
  texcl=\true,
  }

% Гиперссылки в документе
\usepackage[unicode]{hyperref}

% Подключение картинок
\usepackage{graphicx}
% Подписи ко всему
\usepackage{caption}
% ВАЖНО отступы от края 
\voffset=2mm
\RequirePackage{enumitem}
%\textwidth=18cm
%\textheight=235mm
%\hoffset=-20mm
%voffset=-15mm
\setenumerate[1]{fullwidth }

% Переопределяем некоторые названия 
\addto\captionsrussian{\def\contentsname{ОГЛАВЛЕНИЕ}}
\addto\captionsrussian{\def\bibname{\MakeUppercase{Список литературы}}}
\renewcommand\listingscaption{Листинг}
\renewcommand\theFancyVerbLine{\small\arabic{FancyVerbLine}}
\addto\captionsrussian{%
  \renewcommand{\figurename}{Рисунок}%
  \renewcommand{\tablename}{Таблица}%
}

% Переопределяем нумерацию в списке литературы 
\makeatletter
\renewcommand{\@biblabel}[1]{#1.\hfil}
%\usepackage{titlesec}

% Переопределяем оглавление
\usepackage[titles]{tocloft}
\cftsetpnumwidth{3mm}
\renewcommand\cftsecdotsep{\cftdot}
\renewcommand{\cftchappresnum}{ГЛАВА~}
\renewcommand{\cftchapleader}{\cftdotfill{\cftdotsep}}
\renewcommand{\cftchapaftersnum}{.}
\cftsetindents{chapter}{0mm}{5mm}
\newlength{\zyvseclen}
\settowidth{\zyvseclen}{\cftchappresnum\cftchapaftersnum}
\addtolength{\zyvseclen}{2mm}
\addtolength{\cftchapnumwidth}{\zyvseclen}

% Переопределяем вид заголовков разного уровня

\usepackage {titlesec}

\titleformat{\chapter}{\thispagestyle{fancy}\centering\hyphenpenalty=10000\normalfont\normalsize\bfseries}{
ГЛАВА \thechapter. }{0cm}{\normalsize\bfseries}

\titleformat{name=\chapter,numberless}[display]
  {\thispagestyle{fancy}\centering\hyphenpenalty=10000\normalfont\normalsize\bfseries}{}{0cm}{\normalsize\bfseries}
  
  
\titleformat{\section}{\thispagestyle{fancy}\centering\hyphenpenalty=10000\normalfont\normalsize\bfseries}{
\thesection. }{0pt}{\normalsize\bfseries}


\titleformat{\subsection}{\thispagestyle{fancy}\centering\hyphenpenalty=10000\normalfont\normalsize\bfseries}{
\thesubsection. }{0pt}{\normalsize\bfseries}


\titlespacing{\chapter}{0}{0cm}{1.5pt}
\titlespacing{name=\chapter,numberless}{}{}{1.5pt}
\titlespacing{\section}{0}{1.5pt}{1.5pt}
\titlespacing{\subsection}{\parindent}{1.5em}{1em}
\titlespacing{\subsubsection}{\parindent}{1em}{1em}


\bibliographystyle{gost2008}
\usepackage[parentracker=true,
backend=biber,
hyperref=false,
bibencoding=utf8,
style=numeric-comp,
language=auto,
autolang=other,
citestyle=gost-numeric,
defernumbers=true,
bibstyle=gost-numeric,
sorting=ntvy,
]{biblatex}
\bibliography{library.bib}






%Стиль колонтитулов страницы
% Расположение номера страницы


\usepackage{fancyhdr}
\pagestyle{fancy}
\lhead{}
\chead{}
\rhead{}
\lfoot{}
\cfoot{} 
\rfoot{\normalsize\thepage}
\renewcommand{\headrulewidth}{0pt}
\renewcommand{\footrulewidth}{0pt}



\renewcommand{\section}{\@startsection{section}{1}{0pt}%
                                {-0.5ex plus -1ex minus -.2ex}%
                                {0.3ex plus .2ex}%
{\centering\hyphenpenalty=10000\normalfont\normalsize\bfseries}}



\usepackage{enumitem}
\makeatletter
\AddEnumerateCounter{\asbuk}{\@asbuk}{м)}
\makeatother
\setlist[enumerate,itemize]{leftmargin=0pt,itemindent=2.5em}
\setlist{nolistsep}
\renewcommand{\labelitemi}{-}
\renewcommand{\labelenumi}{\asbuk{enumi})}
\renewcommand{\labelenumii}{\arabic{enumii})}



\newcommand{\normI}{\renewcommand{\baselinestretch}{1.}}
\newcommand{\biggI}{\renewcommand{\baselinestretch}{1.1}}
\biggI

%понижает требования к заполнению строк
\sloppy

\usepackage{setspace}
%\полуторный интервал
\onehalfspacing





\begin{document}


% Оформление титульного листа
\begin{titlepage}
\begin{center}
{ \normalsize \singlespacing Министерство образования, науки и молодежной политики 

Нижегородской области 

Государственное бюджетное образовательное учреждение высшего образования 

«Нижегородский государственный инженерно-экономический университет»

(ГБОУ ВО НГИЭУ)
\\[1mm]
}
\begin{flushleft}
\hfill \break
\normalsize{Институт: «Информационные технологии и системы связи»}

\normalsize{Кафедра: «Информационные системы и технологии»}\\[23mm]
\end{flushleft}

\textbf{\Large Отчет \\[1mm] по преддипломной практике\\[26mm]
}
\end{center}

\hfill
\begin{minipage}{.5\textwidth}
\begin{flushright}

Выполнил студент:

института ИТиСС 

очной группы обучения 

4 курса, группы 15ИО 

Капанов А.~А. 

\underline{\hspace{1in}}\\[5mm]

Руководитель практики от кафедры:\\[1mm] 
\underline{\hspace{3cm}}
\hspace{3mm}
Сорочинский А.~И.\\[-1.5pt]
(подпись) \hspace{3.5cm}
(ФИО)
\end{flushright}
\end{minipage}%
\vfill
\begin{center}
 Княгинино 
 
 2019 г.
\end{center}
\end{titlepage}
\newpage

\tableofcontents

\newpage

\chapter*{ВВЕДЕНИЕ}
\addcontentsline{toc}{chapter}{ВВЕДЕНИЕ}
Повышение использования информационных технологий в самых разных отраслях жизнедеятельности человека повлекло за собой сначала малое, а затем все более и более увеличивающеюся число информационных каналов, а следовательно и объема трафика передающегося через него.  

Глобальная сеть Интернет в настоящий момент перенаправляет огромные массивы информации, которые сложно не только отобразить в человеко-читаемом виде в интерфейсах информационных систем, но так же сложно даже просто представить. 
По данным Cisco обьем передаваемой в интернете информации в 2019 году составит около 2 зеттабайт \cite{cisco2018cisco} и практически среди всей этой информации необходимо иметь четкие ориентиры, методы их обработки и отображения. 

Для этого необходимы функциональные поисковые системы, а так же качественные интерфейсы между человеком и компьютером. Причем не только качественные, но и кросплатформенные интерфейсы работающие на любых устройствах. Ранее используемые интерфейсы, стали менее востребованы в связи с переходом большого количества пользователей в мобильный сегмент. 

Популярность стали набирать интерактивные диалоговые системы как компромисс  между информативностью ответа и удобством выражения между устройством и человеком. 

Такие системы предназначены для эффективной обработки сообщений от пользователя на естественном языке, как в аудио формате, так и в текстовом виде. Диалоговые системы ранее строившиеся исключительно на основе правил и онтологий знаний, с развитием машинного обучения и глубокого обучения, получили еще одну сторону для развития в виде имплементаций математических нейронных сетей в различные части архитектуры. Это повысило гибкость систем и увеличило показатели таких систем на порядки. 

Обработка естественного языка (Natural Language Processing) активно развивающаяся сейчас область, которая помогает в построении диалоговых систем.

Благодаря прорывам в данной области появилась возможность разрабатывать практически встраиваемые решения на базе диалоговых систем, своеобразных, «виртуальных помощников» практически для любых предприятий и сфер деятельности. 

Предметом исследования выпускной квалификационной работы являются информационные системы для обработки естественного языка с элементами экспертных систем. 

Объектом исследования – интерактивная информационная система для генерации диалогов на естественном языке. 

Целью данной выпускной квалификационной работы является создание интерактивной диалоговой системы, которая сможет распознавать как письменную, так и устную речь используя статические и нейросетевые методы для ее обработки в задах психологии.

Для достижения поставленной цели в ходе работы необходимо решить ряд задач, таких как: 
\begin{itemize}
\item изучение принципов работы диалоговых систем;
\item анализ типов архитектур и методов реализации;
\item проектирование информационной системы с интеграцией целевого домена знаний; 
\item выбор программных платформ и средств для реализации;
\item реализация информационной системы;
\item апробация.
\end{itemize}

В настоящее время не существует подобных решений в целевой области (психология) на русском языке. Существующие зарубежные решения реализуют только возможности снятия симптомов депрессии и функции систем мониторинга психологического состояния с обратной связью. 

Проектируемая система актуальна, поскольку ее использование предполагает русскоязычную аудиторию, а так же домен применения не рассматривались ранее для проектирования информационных систем. 

\chapter{Описание предметной области}
Диалоговая система – это компьютерная система, предназначенная для общения с человеком, с когерентной (последовательной) структурой \cite{jurafsky2017dialog}.

Первой системой подобного рода считается ELIZA разработанная в MIT в 1966 году. Она реализована при помощи набора правил ответа и поиска во входящей реплике(pattern matching). Такая архитектура называется  <<основанной на правилах>>. ELIZA имитировала диалог с психологом на основе правил активного слушания. В теории это должно было помочь пользователю снять психологическую напряженность или решить несложные психологические неурядицы.  В результате чего ELIZA является представителем «задаче ориентированных диалоговых систем» (Goal based). 

Значительный прогресс в диалоговых системах приходится на 1980-х годах, с ростом популярности и исследованности области машинного обучения и адаптивных алгоритмов. А так же применение новых подходов в компьютерной лингвистке (корпусная семантика, вероятностные грамматики)  позволило начать эксперименты с  алгоритмами машинного обучения в диалоговых системах. Открытие и имплементация символьных грамматик и представлений (Cache Language Models) \cite{li2018recurrent} позволило использовать слова не встречавшиеся раннее в корпусе, облегчив обработку и составление корпусов для обучения. 

Далее исследователи столкнулись с проблемами производительности вычислительных машин того времени и многие подходы до настоящего времени не могли быть использованы. 

В 1990-х в связи с ростом мощности вычислительных устройств стало возможным использование нейросетевого подхода и больших корпусов для обучения. 

До 2006 года об успехах компьютерной лингвистики и диалоговых систем знали только люди интересующиеся, пока компания IBM не устроила показательный матч между системой Watson и человеком в игре-викторине «Jeopardy!» (русская адаптация «Своя игра») , где человек не смог одержать верх над машиной. Тогда общественный интерес к диалоговым системам возрос. В дальнейшем это повлекло масштабные инвестиции в эту область и послужило причиной появления многих  популярных диалоговых агентов. 

В настоящее время большинство диалоговых систем нацелены на мобильные платформы и представляют собой неспециализированных диалоговых ассистентов. Такие ассистенты могут вести беседы на общие темы, использовать возможности устройства по запросу пользователя, а также производить поисковый запрос. 

Примерами таких неспециализированных систем могут служить: 
\begin{itemize}
\item Яндекс.Алиса – персональный помощник для мобильных платформ. Может вести диалог на отвлеченные темы на русском языке, осуществлять поисковые запросы, а так же запускать и взаимодействовать с приложениями установленными на устройстве. Имеет удобный режим диалога для взаимодействия. Сохраняет контекст предыдущих сообщений для формирования ответа. Может отвечать на вопросы вида «одну сущность назад», когда пользователь спрашивать о чем-то о чем говорил ранее \cite{fonarev2017riemannian}. 
\item Siry – персональный помощник от компании Apple. Может вести диалог на нескольких языках, осуществлять поисковые запросы. В целом направлен на взаимодействие с экосистемой устройств Apple. Не сохраняет контекст реплик. 
\end{itemize}
С ростом популярности облачных сервисов многие компании запустили так же сервисы для создания специализированных диалоговых систем на базе своих технологий. 
Примерами таких систем могут служить:
\begin{itemize}
    \item Dialogflow – платформа для создания диалоговых систем от Google. Разворачивается на базе облачного сервиса, поддерживает интеграцию в приложения. Не требует специализированных знаний для работы.
    \item Azure Bot Service – платформа для создания диалоговых систем от Microsoft. Разворачивается на базе облачного сервиса. Требует специальных знаний для оператора.
    \item Amazon Lex – платформа для создания диалоговых систем от Amazon. Разворачивается на базе облачного сервис. Требует специальных знаниий, поддерживает интеграцию с сервисами Amazon. Не доступен на русском языке. 
\end{itemize}
 
 


\chapter{Реализация серверной части}
\section{Проектирование нейросетевой части приложения}
В рамках разделенной структуры сервисов для реализации необходимого функционала системы было принято решение написать функциональные интерфейсы для обученной нейросетевой модели. 

Подход к имплементации архитектуры сети позволял выбирать из множества различных решений. Многие из типовых архитектур позволяют строить не только отображения множества реплик в множества ответов, но также строят языковую модель и семантическое ядро текста. Полнофункциональные модели различных архитектур нейронных сетей имеют множество параметров, что в данном случае является избыточным. 

Было принято решение имплементировать реккурентные типы нейронных сетей, используя типовые архитектуры для генерации по примеру зарубежных коллег \cite{li2016deep, sharma2016natural,толкачев2019нейронное}. Использованные ими различные походы к языковому моделирования диалогов помогли с выбором архитектуры ячеек сети и подготовкой набора данных для обучения. 

Для реализации была выбрана двунаправленная сеть долгой-короткой памяти (bidirectional Long-Short Term Memory network). Языком для реализации послужил Python 3 и набор необходимых библиотек для обучения нейронных сетей.

Для успешного обучения сети необходим качественный и хорошо подготовленный, обширный набор данных. Для целевого домена сети (психология) на момент написания выпускной квалификационной работы открытых наборов данных не существовало. Более того, существующие наборы данных на русском языке для более широкого семантического домена (диалоги) также не позволяли провести успешное обучение диалогового агента. 

Обучение сети проводилось на базе сервиса Colaboratory от компании Google. Сервис предоставляет бесплатные вычислительные мощности для исследователей и студентов в рамках учебных проектов. Colaboratory имеет удобный интерфейс и позволяет запускать фоновые задачи для обучения нейросетевых моделей. Испанские исследователи более подробно описывают функционал сервиса и проводят анализ и сравнение с аналогами \cite{carneiro2018performance}.

Перед обучением сети необходимо подготовить набор данных. Поскольку набор данных необходимо еще было собрать, было принято решение использовать в качестве источника диалогов между двумя коммуникационными агентами смешанный набор из выделенных частей субтитров с сайта Open Subtitles, как это было сделано в работах \cite{creutz2018open, arcan2016asistent} и выделенных из текстов пьес диалогов из открытых источников. 

Перед тем как добавить диалоги из субтитров необходимо было выделить их из файлов субтитров. Для этого был написан скрипт автоматизации экстракции содержимого субтитров. Скрипт извлекает и сортирует содержимое на токены, в данном случае токенами служат приложения. После чего они очищаются от непечатных и иных не поддерживаемых символов, стоп слов. Пример листинга скрипта представлен на листинге \ref{listing:1}, в целом скрипт представляет собой простой класс-обработчик вызываемый при запуске скрипта. 

\begin{listing}[H]
\inputminted[breaklines, breakanywhere, linenos, fontsize=\small]{python}{source/prepare_subtitles.py}
\caption{Часть скрипта обработки субтитров}
\label{listing:1}
\end{listing}

После пред-обработки субтитров, файл отправляется на токенизацию вида <<Вопрос – Ответ>>. Очищенные данные разбиваются на строки и распределяются в два файла. Следующим этапом является разбиение на обучающую и тестовую выборки и выравнивание входящих последовательностей. Листинг кода функций выравнивания представлен на листинге \ref{listing:2}. На этом этапе возможно построение гистограммы длин всех последовательностей ля визуального анализа. Гистограмма размеров представлена на рисунке \ref{fig:len}. 

\begin{listing}[H]
   \inputminted[breaklines, breakanywhere, linenos, fontsize=\small]{python}{source/padding.py}
\caption{Функции реализующие выравнивание входящих последовательностей}
\label{listing:2}
\end{listing}

Далее необходимо преобразовать токены слова сначала в мешок слов. Процесс преобразования включает в себя построение индекса слов, их числового представления распределенного по частоте встречаемости \cite{zhao2017bagwords}. Сформировав индекс возможно построение модели word2vec отображающей множество слов в множество их векторных представлений.

Word2Vec — это набор алгоритмов для расчета векторных представлений слов \cite{word2vecproject}.Он принимает текстовый корпус в качестве входных данных и, после получения словаря из обработанных текстовых материалов, образует векторы слов на выходе. Принцип работы состоит в нахождении связей между контекстами слов, ведь слова, находящиеся в похожих контекстах, часто могут быть семантически близкими. То есть нужно максимизировать косинусную близость между векторами слов, появляющихся в близких контекстах, и минимизировать косинусную близость слов, не появляющихся в контексте друг друга \cite{word2vecproject}. 

При обучении нейронной сети одним из подходов является <<Skip-gram>>, где каждое слово заменяется номером его семантической группы, что позволяет получить <<мешок>> слов с глубоким смыслом, избегая потери учета семантики слова — предсказывается контекст при данном слове. Эти взвешенные векторы фиксированной длины для каждого слова объединяются, используя алгоритм кластеризации (k-средних) \cite{mikolov2013distributed, левченко2017разработка}.

\begin{figure}[H]
    \centering
    \includegraphics[width=0.8\textwidth]{image/hist_len.png}
    \caption{Гистограмма длин }
    \label{fig:len}
\end{figure}

Готовые векторные представления подаются на вход нейронной сети для обучения. Как уже было сказано ранее сеть имеет архитектуру <<bidirectional LSTM>>. Двунаправленные рекуррентные нейронные сети (bi-LSTM) \cite{hochreiter1997lstm} были разработаны для кодирования каждого элемента в последовательности с учетом левого и правого контекстов, что делает их одним из лучших вариантов для решения задачи распознавания именованных сущностей. 

Двунаправленный расчет модели состоит из двух этапов: прямой слой вычисляет представление левого контекста, и обратный слой вычисляет представление правого контекста. Выходы этих шагов затем объединяются для получения полного представления элемента входной последовательности. Было показано, что bi-LSTM полезны во многих задачах естественного языка, таких как машинный перевод, ответ-вопросные системы и особенно в распознавании именованных сущностей. Для реализации интересен именно аспект вопросно-ответных систем \cite{распознаваниесущностей}. 

Для имплементации архитектуры представленной на рисунке \ref{fig:lstm} использовалась библиотека Keras \cite{keras}. Библиотека предоставляет обширные функциональные возможности для проектирования самых разных нейронных сетей, а так же утилиты для подготовки данных, сортировки и другие возможности. Для упрощения проектирования сети использовалась библиотека seq2seq \cite{Britz:2017} позволяющая импортировать уже готовые типовые структуры ячеек и сосредоточится на настройке поток данных и гиперпараметрах. Обучение сети длилось более 10 часов на мощностях сервиса Colaboratory. 

\begin{figure}[H]
    \centering
    \includegraphics[width=0.8\textwidth]{image/bi-LSTM.png}
    \caption{Архитектура bi-LSTM}
    \label{fig:lstm}
\end{figure}

Пример листинга кода построения модели сети на Keras приведен на листинге \ref{listing:3}. Построение модели представляет собой последовательное дополнение структурного графа вычислений который перед исполнением передается библиотеке Tensorflow для перевода в оптимизированный код на языке C.  

\begin{listing}[H]
\inputminted[breaklines, breakanywhere, linenos, fontsize=\small]{python}{source/keras.py}
\caption{Построение архитектуры модели нейронной сети}
\label{listing:3}
\end{listing}

Настроив цикл обучения на тестовый вывод каждой сотой эпохи обучения, получим следующий вывод на этапах обучения, представленный на рисунке . В вывод отображается эпоха, текущая батч-свертка, время, величина потери сети и выходные значения некоторого тестового примера из выборки. Вывод консоли представлен ниже:


\begin{minipage}{0.9\textwidth}
   
        \begin{minted}[breaklines, breakanywhere, fontsize=\small]{pycon}
        Training epoch: 4000, training examples: 10626 - 21252
        Epoch 1/1
        82267/82267 [==============================] - 734s 9ms/step - loss: 0.1241
        . что что это значит не делает ? end end что сейчас ? end end что они ? end end что end все end end всё это end end что 
        \end{minted}
    
\end{minipage}\\[1.5pt]


Постепенно нейронная сеть стремится к уменьшению значения потерь пока они не дойдут до указанного минимума, в данном случае установлено, что \mintinline{python}{ ad = Adam(lr=0.00005) } это значит обучение закончится тогда, когда значение потерь станет равным указанному числу. 

После обучения сети необходимо было написать функции-интерфейсы к объекту модели для использования в приложении Django. В результате обучения сеть способна вести диалоги с контекстной длиной в одну фразу и готова к использованию в проектах. 

\section{Проектирование приложения на фреймворке Django}
Один из самых простых и незамысловатых способов создать веб-приложение на Python с нуля – это воспользоваться стандартом Common Gateway Interface (CGI), который приобрел популярность примерно в 1998 году. Но с развитием веб-стандартов и протоколов обычное приложение CGI уже не удовлетворяет требованиям пользователей. 

Возникает необходимость в гибком подходе к проектированию функциональных веб-приложений. Фреймворк Django спроектирован для решения таких задач. Приложения для фреймворка проектируются по паттерну Модель-Отображение-Контроллер (MVC) при котором за различные части приложения отвечают различные части программного кода. 

Модели представляют объекты которые хранятся в базе данных приложения. В файле <<models.py>> производится перечисление всех абстракций объектов с которыми необходимо работать в хранилище и в классе каждого объекта определяются поля объекта и отношения между ними. На листинге 2.\ref{listing:4} представлен пример объекта-модели проектируемого приложении. 

\begin{listing}[H]
\begin{minted}[breaklines, breakanywhere, linenos, fontsize=\small]{python}
class Message(models.Model):
    """
    Класс экземпляра  сообщения в чате
    """
    id_chat = models.ForeignKey(Chat, on_delete=models.CASCADE)
    date = models.DateTimeField(auto_now_add=True)
    text = models.TextField()
    id_user = models.ForeignKey(User, on_delete=models.CASCADE)

    def __str__(self):
        return str(self.id)

\end{minted}
\caption{Пример модели приложения}
\label{listing:4}
\end{listing}

Внутри класса объекта представлены несколько полей. Два из них являются зависимыми к другим таблицам по типу <<многие-к-одному>> к другим моделям. Ключевой аргумент \mintinline{python}{on_delete = models.CASCADE} позволяет удалять все связанные записи каскадно. Другие поля представляют собой поля типа <<дата/время>> и поле для хранения текста без форматирования. 

В комплекте Django есть модуль Django ORM который реализует работу с различными базами данных, генерируя запросы, таблицы и реляций в полуавтоматическом режиме. 

Настройки базы данных, как и многие другие настройки приложения Django определяются в файле <<settings.py>>. Ниже представлен фрагмент листинга файла приложения: 

\begin{minipage}{0.9\textwidth}
        \begin{minted}[breaklines, breakanywhere, fontsize=\small]{python}
        # Database
        # https://docs.djangoproject.com/en/2.1/ref/settings/#databases
        
        DATABASES = {
            'default': {
                'ENGINE': 'django.db.backends.sqlite3',
                'NAME': os.path.join(BASE_DIR, 'db.sqlite3'),
            }
        }
        \end{minted}
\end{minipage}\\[1.5pt]

Для окружения разработчика используется СУБД <<SQLite>>. Это легковесная встраиваемая СУБД в формате C-библиотеки позволяет выстраивать окружение разработчика без установки дополнительных инструментов управления конфигурацией СУБД.

Структура базы данных приложения представлена на рисунке \ref{fig:db}.

\begin{figure}[H]
    \centering
    \includegraphics[width=0.9\textwidth]{image/db_diagram.png}
    \caption{Структура базы данных приложения}
    \label{fig:db}
\end{figure}

Django позволяет создавать веб-приложение, но в соответствии со структурой проектируемого приложения необходимо предоставить клиентской части API для работы. В качестве стандарта используется стандарт соглашения о REST структуре \cite{richardson2008restful}. Для реализации  REST-апи используется библиотека <<Django REST Framework>> (DRF). 

Для работы DRF необходимо создать файлы сериализации моделей, а также файл в котором будут отображены правила взаимодействия и вызываемые функции в процессе работы с интерфейсами приложения. 

Ниже приведен пример листинга файла <<serializer.py>>:

\begin{minipage}{0.9\textwidth}
        \begin{minted}[breaklines, breakanywhere, fontsize=\small]{python}
        from rest_framework import serializers

        from psiChat.models import Dialogue
        
        
        class DialogueSerializer(serializers.HyperlinkedModelSerializer):
            class Meta:
                model = Dialogue
                fields = ('id','name', 'description', 'number_steps',)

        \end{minted}
\end{minipage}\\[1.5pt]

В начале скрипта импортируется модель которой необходим интерфейс для сериализации в JSON и обратно. После чего создается класс наследуемый от класса <<HyperlinkedModelSerializer>>. В мета-классе, который при выполнении интерпретатор преобразует в свойства класса и вызовы функций, определяем поля которые необходимо сереализовать. В примере представлен типовой сериалализатор для модели который предоставляет готовые автоматически сгенерированные функции для различных вызовов. 

После создания сериализатора для работы приложения необходимо создать ее представление. Для этого рассмотрим скрипт <<dialigue.py>> (листинг 2.\ref{listing:5}) в котором определено представление модели через типовой класс DRF -- ModelViewSet. В начале скрипта импортируем необходимые объекты и функции. Определяем размер выборки из базы данных при помощи свойства класса <<queryset>>. Так же определим классы доступа к представлению модели при помощи свойства <<permissions\_classes>>, для всех пользователей. 

\begin{listing}[H]
\begin{minted}[breaklines, breakanywhere, linenos, fontsize=\small]{python}
    from rest_framework import viewsets
    import rest_framework.permissions as permissions
    from psiChat.models import Dialogue
    from psiChat.serializers import DialogueSerializer
    
    
    class DialogueViewSet(viewsets.ModelViewSet):
        queryset = Dialogue.objects.all()
        serializer_class = DialogueSerializer
        permissions_classes = [permissions.AllowAny]
        def get_queryset(self):
            """
            This view should return a list of all records
            """
            return Dialogue.objects.all().order_by('name')

\end{minted}
\caption{Пример представления модели для API}
\label{listing:5}
\end{listing}

Закончить подготовку API можно сформировав необходимые сериализаторы и представления и далее просто определить их адреса через типовой класс <<router>> предоставляемый DFR. 

\section{Интеграция модели Tensorflow в приложение Django}
Следующей задачей необходимой для реализации программы является интеграция сформированной в пункте 2.1 модели нейронной сети. 

Существующие решения (например, <<TensorFlow Serve>>) для интеграции не предоставляют достаточной гибкости и требуют обслуживания и развертывания дополнительных сервисов для работы. Было принято решение интегрировать модель как модуль для вызова во время запуска Django-приложения. 

Для выполнения этой задачи необходимо произвести специализированную настройку графа вычислений при запуске Django. Во время запуска необходимо произвести <<заморозку>> и импортирование единственного графа вычиcлений. Объявим переменные хранящие путь до сохраненной обученной модели и векторным представлениям слов. Реализация этой задачи представлена ниже: 

\begin{minipage}{0.9\textwidth}
        \begin{minted}[breaklines, breakanywhere, fontsize=\small]{python}
        filename_net_model = os.path.join(BASE_DIR, 'psiChat/chatModule/chitchat/data/net_model.txt')
        filename_net_weights = os.path.join(BASE_DIR, 'psiChat/chatModule/chitchat/data/net_final_weights.h5')
        filename_w2v_model = os.path.join(BASE_DIR, 'psiChat/chatModule/chitchat/data/w2v_model.bin')
        pr = Prediction(filename_net_model=filename_net_model,
                        filename_net_weights=filename_net_weights,
                        filename_w2v_model=filename_w2v_model)
        global graph
        graph = tensorflow.get_default_graph()
        MODEL = pr
        \end{minted}
\end{minipage}\\[1.5pt]

Выполнив все выше перечисленные действия можно спроектировать и запрограммировать серверную часть приложения с интегрированной предобученной моделью Tensorflow.




\chapter{Программирвоание клиентской части}
\section{Web-socket в клиентской части}

Сложные веб-приложения, обладающие насыщенными динамическими пользовательскими интерфейсами, воспринимаются как нечто само собой разумеющееся. А ведь интернету пришлось пройти долгий путь для того, чтобы достичь его сегодняшнего состояния.

В самом начале интернет не был рассчитан на поддержку подобных приложений. Он был задуман как коллекция HTML-страниц, как «паутина» из связанных друг с другом ссылками документов. Всё было, в основном, построено вокруг парадигмы HTTP «запрос/ответ». Клиентские приложения загружали страницы и после этого ничего не происходило до того момента, пока пользователь не щёлкнул мышью по ссылке для перехода на очередную страницу.

Примерно в 2005-м году появилась технология AJAX \cite{garrett2005ajax} и множество программистов начало исследовать возможности двунаправленной связи между клиентом и сервером. Однако, все сеансы HTTP-связи всё ещё инициировал клиент, что требовало либо участия пользователя, либо выполнения периодических обращений к серверу для загрузки новых данных.

Технологии, которые позволяют «упреждающе» отправлять данные с сервера на клиент существуют уже довольно давно. Среди них — Push и Comet.

Один из наиболее часто используемых приёмов для создании иллюзии того, что сервер самостоятельно отправляет данные клиенту, называется «длинный опрос» (long polling). С использованием этой технологии клиент открывает HTTP-соединение с сервером, который держит его открытым до тех пор, пока не будет отправлен ответ. В результате, когда у сервера появляются данные для клиента, он их ему отправляет.

Спецификация WebSocket определяет API для установки соединения между веб-браузером и сервером, основанного на «сокете». Это постоянное соединение между клиентом и сервером, пользуясь которыми клиент и сервер могут отправлять данные друг другу в любое время.

Клиент устанавливает соединение, выполняя процесс так называемого рукопожатия WebSocket. Этот процесс начинается с того, что клиент отправляет серверу обычный HTTP-запрос. В этот запрос включается заголовок Upgrade, который сообщает серверу о том, что клиент желает установить WebSocket-соединение.

URL, применяемый для WebSocket-соединения, использует схему ws. Кроме того, имеется схема wss для организации защищённых WebSocket-соединений, что является эквивалентом HTTPS.

Внутри клиентской части приложения используется библиотека <<vue-socket-io>>. На листинге 3.\ref{listing:6} представлено подключение библиотеки к фреимфорку Vue и использование ее совместно с библиотекой состояния Vuex.

\begin{listing}[H]
\begin{minted}[breaklines, breakanywhere, linenos, fontsize=\small]{javascript}
import Vue from 'vue'
import store from '../store'
import VueNativeSock from 'vue-native-websocket'
Vue.use(VueNativeSock, 'ws://25.54.121.49:8000/ws/', {format: 'json', store: store});
\end{minted}
\caption{Использование библиотеки vue-socket-io в javascript коде}
\label{listing:6}
\end{listing}

После подключение к библиотеке можно обращаться изменяя состояния приложение при помощи библиотеки Vuex \cite{halliday2018vue}. На листинге 3.\ref{listing:7} представлено использование библиотеки в контроллере состояний Vuex. 

Vuex выполняет роль хранилища состояний приложения обеспечивая консистентные изменения и <<источник истины>> для всего приложения. Использования менеджера состояний позволяет лучше конролировать потоки и состояний данных, соединений и контексты выполнения \cite{halliday2018vue}.

Websocket позволяет обеспечить консистентность состояний между сервером и клиентом. На серверной стороне используется библиотека Django Channels для реализации соединения с приложением Django.

При работе с вебсокетом переход к различным состояниям хранилища приложения происходит при вызове мутаций (mutation) в контекстах действий (actions) которые атомарно изменяют состояние целиком. Это предотвращает ошибки потери запросов и данных.  

\begin{listing}[H]
   \inputminted[breaklines, breakanywhere, linenos, fontsize=\small]{javascript}{source/vue-socket.js}
\caption{Использование vue-socket-io в связке с Vuex}
\label{listing:7}
\end{listing}


\section{Реализация интерфейса клиентской части}

Клиентская часть написана при помощи стандартных средств разметки, а так же фреймворка Vue в связке с микрофреймворком  <<Vuetify>>. 

Структура организации приложения (рисунок \ref{fig:file_tree}) предполагает разделение его на компоненты и представления согласно шаблонам Vue. Внутри содержатся однокомпонентные файлы  с расширением ".vue" совмещающие в себе как разметку компонента, так и js-код и каскадные стили. 

\begin{figure}[H]
    \centering
    \includegraphics[width=0.3\textwidth]{image/fileTree.png}
    \caption{Файловая структура клиентской части}
    \label{fig:file_tree}
\end{figure}

Типовой файл vue содержит три семантические единицы: 

\begin{itemize}
    \item разметка структуры компонента \mintinline{html}{<template></template>},
    \item программная часть компонента \mintinline{html}{<script></script>},
    \item стили компонента \mintinline{html}{<style></style>}.
\end{itemize}

После написания компонента требуется транспиляция кода в нативный js-код и разметку. 

В качестве сборщика использовался <<Webpack>>. Пример файла компонента представлен на листинге \ref{listing:8}. 

\begin{listing}[H]
   \inputminted[breaklines, breakanywhere, linenos, fontsize=\small]{html}{source/vue_comp.vue}
\caption{Представление главного экрана клиентской части приложения}
\label{listing:8}
\end{listing}

Фреймворк Vuetify предоставляет обширную коллекцию стилизованных компонентов, с возможностью переопределения и доработки. Главный экран приложения представлен на рисунке \ref{fig:main_page}. 
\begin{figure}[H]
    \centering
    \includegraphics[width=0.9\textwidth]{image/main_page.png}
    \caption{Главная страница приложения}
    \label{fig:main_page}
\end{figure}

На главном экране представлены компоненты чата, переход на страницу администратора и кнопка авторизации. В главном окне пользователь производит общение с чат-ботом. 

При переходе к странице администратора открывается следующая страница представленная на рисунке 



\chapter*{Заключение}
\addcontentsline{toc}{chapter}{Заключение}

В результате спроектированная  информационная систем система реализует большую часть требуемого функционала. Запланированные задачи достигнуты в процессе реализации, такие как: 

\begin{itemize}
\item изучение принципов работы диалоговых систем на примере архитектур диалоговых систем;
\item анализ типов архитектур и методов реализации и выбрана клиент-серверная архитектура;
\item проектирование информационной системы с интеграцией целевого домена знаний в виде возможности интеграции набора данных администратором системы;  
\item выбор программых платформ и средств для реализации -- выбраны Python c фреймворком Django для серверной части и EcmaScript для клиентской части с фреймворком Vue.
\end{itemize}




\newpage
\addcontentsline{toc}{chapter}{СПИСОК ИСПОЛЬЗОВАННОЙ ЛИТЕРАТУРЫ}
\printbibliography
\end{document}