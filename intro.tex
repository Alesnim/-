\chapter*{ВВЕДЕНИЕ}
\addcontentsline{toc}{chapter}{ВВЕДЕНИЕ}
Повышение использования информационных технологий в самых разных отраслях жизнедеятельности человека повлекло за собой сначала малое, а затем все более и более увеличивающеюся число информационных каналов, а следовательно и объема трафика передающегося через него.  

Глобальная сеть Интернет в настоящий момент перенаправляет огромные массивы информации, которые сложно не только отобразить в человеко-читаемом виде в интерфейсах информационных систем, но так же сложно даже просто представить. 
По данным Cisco обьем передаваемой в интернете информации в 2019 году составит около 2 зеттабайт \cite{cisco2018cisco} и практически среди всей этой информации необходимо иметь четкие ориентиры, методы их обработки и отображения. 

Для этого необходимы функциональные поисковые системы, а так же качественные интерфейсы между человеком и компьютером. Причем не только качественные, но и кросплатформенные интерфейсы работающие на любых устройствах. Ранее используемые интерфейсы, стали менее востребованы в связи с переходом большого количества пользователей в мобильный сегмент. 

Популярность стали набирать интерактивные диалоговые системы как компромисс  между информативностью ответа и удобством выражения между устройством и человеком. 

Такие системы предназначены для эффективной обработки сообщений от пользователя на естественном языке, как в аудио формате, так и в текстовом виде. Диалоговые системы ранее строившиеся исключительно на основе правил и онтологий знаний, с развитием машинного обучения и глубокого обучения, получили еще одну сторону для развития в виде имплементаций математических нейронных сетей в различные части архитектуры. Это повысило гибкость систем и увеличило показатели таких систем на порядки. 

Обработка естественного языка (Natural Language Processing) активно развивающаяся сейчас область, которая помогает в построении диалоговых систем.

Благодаря прорывам в данной области появилась возможность разрабатывать практически встраиваемые решения на базе диалоговых систем, своеобразных, «виртуальных помощников» практически для любых предприятий и сфер деятельности. 

Предметом исследования выпускной квалификационной работы являются информационные системы для обработки естественного языка с элементами экспертных систем. 

Объектом исследования – интерактивная информационная система для генерации диалогов на естественном языке. 

Целью данной выпускной квалификационной работы является создание интерактивной диалоговой системы, которая сможет распознавать как письменную, так и устную речь используя статические и нейросетевые методы для ее обработки в задах психологии.

Для достижения поставленной цели в ходе работы необходимо решить ряд задач, таких как: 
\begin{itemize}
\item изучение принципов работы диалоговых систем;
\item анализ типов архитектур и методов реализации;
\item проектирование информационной системы с интеграцией целевого домена знаний; 
\item выбор программных платформ и средств для реализации;
\item реализация информационной системы;
\item апробация.
\end{itemize}

В настоящее время не существует подобных решений в целевой области (психология) на русском языке. Существующие зарубежные решения реализуют только возможности снятия симптомов депрессии и функции систем мониторинга психологического состояния с обратной связью. 

Проектируемая система актуальна, поскольку ее использование предполагает русскоязычную аудиторию, а так же домен применения не рассматривались ранее для проектирования информационных систем. 